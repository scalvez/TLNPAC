In this experiment, muons below $500$~MeV loose all their energy in the detectors and decay within it.
The time between the muon and electron signal is then measured to obtain the muon lifetime.
The detection of muons and electrons is made with a $9.5$~cm thick plastic scintillator ($R_0$) with two photomultipliers (PM). 
For trigger purpose, three plastic slates ($R_1$, $R_2$, $R_{-1}$) are also used. 
The geometric organisation of the different detectors is given in \Cref{DetectorLayers}.

The slate $R_2$ is placed at large distance from the main scintillator $R_0$ in order to trigger the most vertical muons.
The slate $R_1$ is placed close to $R_0$ in order to reduce the number of muons reaching $R_0$ without passing through $R_1$.
This will reduce the background of the electron signature.
Finally, $R_{-1}$ is placed under $R_0$ and a layer of lead to detect muons which haven't been stopped. 
The lead is to stop the electrons that may escape from $R_0$.

PMs have been calibrated to low voltage (about $1.6$~kV) in order to keep the inner background below the trigger threshold.
To enter into the trigger selection, a signal must pass a threshold of $30$~mV.
The voltage corresponding to slope rupture in the signal frequency plot have been applied for each PM.
This setup reduces the interesting signal frequency but also suppress the background.

Two logic triggers are tuned to select the muon events and  electron events.
A muon event have to deposit energy in $R_2$, $R_1$ and $R_0$ but must not coincide with any deposit in $R_{-1}$. 
The deposit in $R_0$ must be detected by both multipliers.
This trigger keep vertical muons which stop in the main detector.
When an event pass the muon trigger, a logic gate opens to wait for an electron triggered event.
The electron trigger needs a deposit in $R_0$ but no deposit in any slate. 
This trigger caracterize an event that don't come from the vertical and do not escape in the vertical direction.
The logic diagram of the trigger is given in \Cref{LogicTrigger}
